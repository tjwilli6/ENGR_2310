\documentclass{article}
\usepackage[margin=1.5cm,bottom=2cm]{geometry}
\usepackage{fancyhdr}
\usepackage{graphicx}
\usepackage[section]{placeins}
\pagestyle{fancy}
\usepackage{hyperref}
\usepackage[export]{adjustbox}
\pagestyle{fancy}
\usepackage{xcolor}
\hypersetup{colorlinks=true,urlcolor=blue,urlbordercolor=blue}

\begin{document}
\fancyhead[L]{ \includegraphics[width=2cm]{au_logo.png} }
\fancyhead[R]{ENGR 2310: Computational Problem Solving}
\fancyfoot[C]{\thepage}
\vspace*{0cm}
\begin{center}
	{\LARGE \textbf{Style and Readability Guidelines}}\\
	\vspace{.25cm}
	%{\Large Projectile Motion}
	%\vspace{0.25cm}
	%{\Large Due: Friday, September 4}
\end{center}

\section*{Code Organization}
Your .py file should be organized in the following way:
\begin{itemize}
	\item A commented header stating your name, the date, and a brief overview of the code
	\item Import any modules you will need 
	\item Create any constant, global variables
	\item Define all functions used by your program
	\item Your actual program code, inside a function called \texttt{main()}
	\item The last line in your file is a simple call to \texttt{main()}
\end{itemize}
\section*{Readability}
When possible, pick variable and function names whose meaning is obvious. Organize code into functions when possible. Don't use them for \textit{everything}.
\subsection*{When to use a function}
\begin{itemize}
\item If you find yourself copying and pasting code, definitely use a function!
\item If you think you might reuse that code in the future, even if your current program doesn't, you should use a function
\item If a calculation takes several lines, depends on just a few variables, and results in a single value, that's a good sign that you should use a function
\end{itemize}
\section*{Comments and Doc strings}
As a general rule, write comments as if someone else (who knows Python, but has never met you) needs to understand your program.
This includes:
\begin{itemize}
	\item Commenting when introducing a new variable (unless the variable name makes it obvious, like \texttt{PI=3.14})
	\item Commenting to explain the harder-to-understand parts of your code (what is this if statement doing? Why are we looping here?)
\end{itemize}
\section*{Example}
See an example file that follows these guidelines \href{https://drive.google.com/file/d/1GPYnWsq01MTotlsTjylKzifQ6dDCEbD2/view?usp=sharing}{here}
\end{document}