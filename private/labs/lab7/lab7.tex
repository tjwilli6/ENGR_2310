\documentclass{article}
\usepackage[margin=1.5cm,bottom=2cm]{geometry}
\usepackage{fancyhdr}
\usepackage{graphicx}
\usepackage[section]{placeins}
\usepackage{xcolor}
\usepackage{amsmath}
\usepackage{hyperref}
\usepackage[export]{adjustbox}
\pagestyle{fancy}

\hypersetup{colorlinks=true,urlcolor=blue,urlbordercolor=blue}
\begin{document}
\fancyhead[L]{ \includegraphics[width=2cm]{../au_logo.png} }
\fancyhead[R]{ENGR 2310: Computational Problem Solving}
\fancyfoot[C]{\thepage}
\vspace*{0cm}
\begin{center}
	{\LARGE \textbf{Lab 7}}\\
	\vspace{.25cm}
	%{\Large Basic Python}
	%\vspace{0.25cm}
	%{\Large Due: Friday, September 4}
\end{center}

\begin{enumerate}
	\item Here is some Python code to create a list of the eight planets:\\ \\
	\texttt{planets = ["Mercury","Venus","Earth","Mars","Jupiter","Saturn","Uranus","Neptune"]}\\
	\begin{enumerate}
		\item Write a line of code to print the fourth planet, which should be ``Mars'' (don't just print ``Mars'', access the element from inside the list)
		\item Write some code to tell you the position of ``Saturn'' within the list
		\item A new planet is discovered! It is named ``Minerva'' (after another Roman god). The planet orbits past Neptune. Write some code to add this planet to the ``planets'' list.
		\item The planet ``Earth'' is to be renamed to ``Terra''. Write some code to make this happen.
		\item ``Mars'' was ``accidentally'' destroyed by Elon Musk. Write some code to remove it from the list.
		\item Write some code to loop over every planet in the list and print its position and name. Example output:\\
		Planet 1 is Mercury\\Planet 2 is Venus\\etc...
	\end{enumerate}
	\item Write a Python program to plot the first $N$ terms of the Fibonacci sequence (term value on the y axis, term number on the x). $N$ is supplied by the user. Make sure to label your axes!
	\item A physics student is measuring the acceleration vs mass for a cart being pulled by a constant force. Here is her data:\\
	\begin{center}
	\begin{tabular}{|c|c|}
		\hline
		Mass [kg]&Acceleration [m$\cdot$s$^{-1}$]\\
		\hline
		0.1&302.6\\
		\hline
		0.2&147.3\\
		\hline 
		0.3&97.6\\
		\hline 
		0.4&72.7\\
		\hline
		0.5&62.8\\
		\hline
		0.6&46.4\\
		\hline 
		0.7&42.5\\
		\hline
		0.8&38.8\\
		\hline
		0.9&34.6\\
		\hline
		1.0&29.3\\
		\hline
	\end{tabular}
	\end{center}
	Make a graph of acceleration vs mass. You should use a scatter plot to show data points like this. Calculate the average acceleration and plot it as a horizontal line using the \href{https://matplotlib.org/stable/api/_as_gen/matplotlib.pyplot.axhline.html}{pyplot.axhline} function
	\item Modify your projectile motion code (homework 2) so that it graphs $x_{max}$ vs $\theta$ (like what I showed in class). Make sure to label your axes. Add a vertical line (you can use \href{https://matplotlib.org/stable/api/_as_gen/matplotlib.pyplot.axvline.html}{pyplot.axvline}) to show which $\theta$ value maximizes $x_{max}$.
\end{enumerate}
\end{document}