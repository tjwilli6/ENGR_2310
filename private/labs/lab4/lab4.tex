\documentclass{article}
\usepackage[margin=1.5cm,bottom=2cm]{geometry}
\usepackage{fancyhdr}
\usepackage{graphicx}
\usepackage[section]{placeins}
\usepackage{xcolor}
\usepackage{amsmath}
\usepackage{hyperref}
\usepackage[export]{adjustbox}
\pagestyle{fancy}

\hypersetup{colorlinks=true,urlcolor=blue,urlbordercolor=blue}
\begin{document}
\fancyhead[L]{ \includegraphics[width=2cm]{../au_logo.png} }
\fancyhead[R]{ENGR 2310: Computational Problem Solving}
\fancyfoot[C]{\thepage}
\vspace*{0cm}
\begin{center}
	{\LARGE \textbf{Lab 4}}\\
	\vspace{.25cm}
	%{\Large Basic Python}
	%\vspace{0.25cm}
	%{\Large Due: Friday, September 4}
\end{center}

\section*{Questions}
\begin{enumerate}
	\item What will this code output? (First predict, then check!)\\
	\texttt{def greet():}\\
	\null\quad\texttt{print("Hello!")}\\
	\texttt{}\\
	\texttt{print("What's up?")}\\
		Access this code \href{https://drive.google.com/file/d/1E3bcdBPJzZ3XdDHAZoqRmL6b_PFArfmh/view?usp=sharing}{here}
	\item Consider the following program:\\
	\texttt{def print\_greetings(name):}\\
	\null\quad\texttt{print("Hello, ",name)}\\
	\texttt{}\\
	\texttt{print\_greetings("Pam")}\\
	Access this code \href{https://drive.google.com/file/d/1datgwaYyWWi5tQ8OYvKllhb2piJZYZbX/view?usp=sharing}{here}
	\begin{enumerate}
		\item What is the argument to the function call? What is the function parameter?
		\item Modify the program to create a variable, \texttt{name = "Pam"}, and send this variable as the argument to the \texttt{print\_greetings} function. Does sending the argument as a variable affect your program?
		\item Modify the program once more. Remove the \texttt{name = "Pam"} statement, and add a \texttt{person = "Pam"}. Now pass \texttt{person} as the argument. Does it matter that the argument variable (\texttt{person}) has a different name than the function parameter (\texttt{name})?
	\end{enumerate}
	\item The programmer was expecting this program to print 200. What does it print instead?\\\texttt{def proc(x):}\\
	\null\quad\texttt{x = 2 * x * x}\\
	\texttt{}\\
	\texttt{def main():}\\
	\null\quad\texttt{num = 10}\\
	\null\quad\texttt{proc(num)}\\
	\null\quad\texttt{print(num)}\\
	\texttt{}\\
	\texttt{main()}\\
		Access this code \href{https://drive.google.com/file/d/1MGlbId_BLjoFRdNz8qp-OS1lGKVOViBQ/view?usp=sharing}{here}
	\item What will be printed by the following program? (First predict, then check!)\\
	\texttt{def addnumbers(num1,num2):}\\
	\null\quad\texttt{print("Sum is: ", num1 + num2)}\\
	\texttt{}\\
	\texttt{addnumbers(2,3)}\\
	\item What will be printed by the following program? (First predict, then check!)\\
	\texttt{def addnumbers(num1,num2):}\\
	\null\quad\texttt{print("Sum is: ", num1 + num2)}\\
	\texttt{}\\
	\texttt{addnumbers(5,7,2)}\\
	\item What will be printed by the following program? (First predict, then check!)\\
	\texttt{def addnumbers(num1,num2):}\\
	\null\quad\texttt{print("Sum is: ", num1 + num2)}\\
	\texttt{}\\
	\texttt{addnumbers(4)}\\
	\item What will this program print? (First predict, then check!)\\
	\texttt{def proc():}\\
	\null\quad\texttt{name = "Toby"}\\
	\texttt{}\\
	\texttt{name = "Taylor"}\\
	\texttt{proc()}\\
	\texttt{print(name)}\\
	Access this code \href{https://drive.google.com/file/d/1C2JdkQ23hZkatLRrj7PbGVye3sl6ZVFJ/view?usp=sharing}{here}
\end{enumerate}


\section*{Writing Programs}
\begin{enumerate}
\item Write a program to print the value of gravitational acceleration $g$ on any planet, given values for the planet's mass and radius. Use a function to perform the calculation. The user should input the mass and radius, and then these values should be passed as arguments to the function.
\begin{equation*}
g=\frac{GM}{R^2}
\end{equation*}
Where $G$ is Newton's constant: $G\approx6.7\times10^{-11}$ N$\cdot$m$^2\cdot$kg$^{-2}$. Verify that $g\approx9.8$ m$\cdot$s$^{-2}$ for Earth ($R\approx6.4\times10^6$ m, $M\approx6\times10^{24}$ kg) \textit{Note: To input numbers this large and have them successfully convert to floats, use ``e notation'', so $6\times10^{24}$ becomes 6e24. You are also free to type a 6 followed by 24 zeros.}

\item Modify your quadratic equation code to use a function, \texttt{solve\_quadratic(a,b,c)} which takes 3 numbers as parameters and prints the solution to the quadratic equation $ax^2+bx+c=0$ (you can and should use the code you've already written, just turn it into a function)
\end{enumerate}

\end{document}